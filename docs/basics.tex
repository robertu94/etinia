\chapter{The Basics}
\section{What is Etinia?}
Etinia is a game engine.  It is designed to be a tool set to facilitate a large
variety of different kinds of games including Role Playing Games and Tactical
Combat Games.  This book contains the core rules for using this system, some
character samples, and the details of a world system designed to work with the
Etinia engine.
\section{A note on roleplaying}
The objective of a role playing game is to collaboratively tell a story between
the players.  For some players, this is best accomplished with heavy rule sets
that clearly delineate skills so that each player is strictly balanced with
every other player.  For others, this is best accomplished with light rule sets
that allow players to do wild and crazy things that are not at all balanced.
This rule set attempts to strike the best of both worlds by creating a modular
rule set.  This means the tactical players have their strict rules set, but
role playing players may simply play without them.  Unfortunately, one of these
styles requires much more precise language in order to achieve clarity.
Therefore, these rules will be written primarily to the audience of tactical
players.  For those of you who prefer a lighter game, simply take the rules and
focus on what kind of abilities each level provides.  For example, a Life mage
who can make a fireball could also light a candle with magic.  The rules may
not clearly delineate this ability but it is obvious from the spirit of the
rules.  This ambiguity then requires some sort of authority.  Any ambiguity is
to be determined solely by the game master.  Furthermore, these rules make
several clear references to distances, for a lighter game consider anything
with in 1 space to be close, 2 spaces to be mid ranged, and 3+ spaces to long
ranged.  Therefore these distance rings could be used instead of a battle map.
Furthermore, for references to width assume that each weapon can hit the number
of enemies equal to the area in spaces of that it could hit on a battle map.
One final recommendation for role playing heavy groups would be for the game
master to assign a number of rp points to each player.  These points could be
used to overrule a judgment call on a rule or to "re-roll" a failed attack or
test.

\section{How to get started}

\subsection{Hero}
The basis of Etinia is the \index{Hero}Hero.  To begin the game, one should
create a Hero.  To create a Hero, one receives 11 skill points and 6 equipment
points.  The skill points are used to purchase skills.  The equipment points
are used to purchase equipment.  Heroes have Attributes, Skills, Abilities,
Equipment, and an Elemental Affiliation.  Attributes are basic statistics that
determine how the character functions at a very basic level.  There are 4 basic
attributes: Will, Strength, Defense, and Speed.  Will is the ability of the
character to remain conscious after taking damage.  It determines how many hits
a player may take before falling unconscious.  Will starts at 6 and may be as
high as 11.  Strength is the ability of the Hero to perform actions.  It is a
factor in every action.  Defense is the ability of a Hero to avoid being hit
and taking damage.  It is a factor used in stopping actions done against the
player.  Speed is the ability of the Hero to move quickly.  It determines when
and how often Heroes move.  Strength, Defense, and Speed start at 0 and may
become as high as 5.  Skills are the ways heroes use their attributes perform
actions in the game, and how heroes can improve their attributes.  Heroes can
have up to 5 skills.  Skills come in 5 skill levels.  Each level is more
powerful than the last and includes the abilities of every previous level.
Level 1 skills are the weakest, and level 5 skills are the most powerful.
Over the course of the game, heroes will level up.  When Heroes level up, they
may acquire new skills and improve their current ones.  However, Heroes cannot
have two skills with the same name.  For example, a hero cannot have Close
Combat 2 and Close Combat 3.  All skills have a price in skill points based on
their level.  


\begin{center}
\rowcolors{1}{white}{gray}
\begin{tabularx}{\textwidth}{X X}
\hline
Skill Level & Cost in Skill Points \\ \hline
1 & 1 \\
2 & 3 \\
3 & 6 \\
4 & 10 \\
5 & 15 \\
\hline
\end{tabularx}
\end{center}

For example, if a player wanted Battle Knowledge 3 and Cast 2 (two different
skills); the player would need 8 skill points.  Skills are organized into Skill
Trees based on function.  There are seven different skill trees:
Attack, Defense, Knowledge, Magic, Mechanics, Ancients and Racial.  Skills in the
Attack Tree allow characters to deal damage to opponents.  These skills are
handy for any damage dealing character or weapons master character.  Improving
the skills allows you to increase the range, deal more hits, and unlock more
options.  The Defense Tree allows player to reduce both directly and indirectly
the damage that they receive.  These skills are useful for any wall or tank
character.  As characters level up these skills, they will be able to greater
reduce the damage they take.  The Knowledge Tree is the house of all knowledge
in the game.  The uses of the skills in the knowledge tree are as diverse as
the knowledge itself.  Each skill contains a variety of different abilities
that become more and more potent as the character advances them.  The Magic
Tree is home to all things dealing with magic.  Characters based off
mages, sages, and other magic casters will benefit greatly from the skills in
this tree.  However with great power comes great cost.  Many skills in this
tree require energy to be expended to be used.  As the skills in this tree are
advanced, the abilities become more and more powerful.  The Mechanics Tree is
where robots programs and machines take their power.  However, with great power
comes great cost.  Many skills in this tree can be prevented when damage is
taken.  As the skills in this tree are advanced, the abilities become more and
more powerful.  The Ancients Tree is home to old power.  Heroes based off
colossal warriors can find their skills here.  These skills give extreme power
for a limited time.  As the skills in this tree are advanced, the abilities
become more and more powerful.  The Racial Tree contains skills that change the
attributes of the heroes who have them.  Skills such as
Strength, Defense, Speed, and Will are located here as well as special racial
abilities.  As these skills are advanced, the attributes that they represent are
improved.  The skills that a Hero has determine which Abilities they may use.
Abilities are specific actions that a player may perform.  For example a
character with Close Combat 1 gains the ability long cut which when equipped
with a short range weapon may target any hero in within two panels in front of
him.  In addition having the necessary skill, Heroes must also choose to ready a
skill.  Readying a skill is simply including the ability in a list of skills
that the hero will use.  Abilities may be readied at any time outside of
combat.  Each hero may have up to 20 abilities readied.

\subsection{Equipment}
\subsubsection{Making equipment}
Heroes also get 5 Equipment Points to use to purchase equipment.  There are several different types of equipment. \\
\rowcolors{2}{white}{gray}
\begin{tabularx}{\textwidth}{X X}
\hline
Equipment Type & Function \\
\hline 
Armor & Protects Hero;  Sets affinity\\
Helm & Protects Hero\\
Footwear & Protects Hero\\ 
Short Range & Use Close Combat attacks\\
Long Range & Use Long Range attacks \\
Staff & Improves Magic Tree attacks\\
Shield & Use Shield abilities\\
Gloves & Improves Melee attacks\\
\hline
\end{tabularx}\\

All equipment is made using the following materials which determine its statistics
\begin{center}
\rowcolors{2}{white}{gray}
\begin{tabularx}{\textwidth}{X X X X}
\hline
Cost& Material& Attack/Defense & Engineer\\
\hline
1 & Organic & 1/1 & 0\\
3& Stone & 2/2 & 0\\
6& Ore& 3/3& 1\\
10& Metal& 4/4& 2\\
15& Processenium& 5/5& 3\\
N/A& Refined Alloy& 6/6& 3\\
N/A& Legendary Alloy& 7/7& 4\\
\hline
\end{tabularx}\\
\end{center}


\subsubsection{Improving equipment}

\paragraph{Engineering}

Engineering is preformed by the Technical skill.  Each material has different
properties that affect the extent to which it can be modified by engineered.
There is no cost to engineering an item during character creation.  The effects
of the modifications are permanent.  Equipment may only be engineered at
creation. \\
\begin{center}
\rowcolors{1}{white}{gray}
\begin{tabularx}{\textwidth}{X X}
\hline
Engineering level & Effect \\

\hline0 & No effect \\
1 & $+1$ attack $-1$ defense or $+1$ defense $-1$ speed or $+1$ speed $-1$ attack \\
2 & 2 Rank 1 or +1 MP \\
3 & 2 Rank 2 or Gain Ability from a Skill Level 1\\
4 & 2 Rank 3 or Gain Level 1 Skill \\
\hline
\end{tabularx}\\
\end{center}

\paragraph{Alchemy}
Alchemy is preformed by the alchemy skill.  All materials may use all levels of
alchemy.  The effects of alchemy are permanent.  Characters with the alchemy
skill do not have to pay the cost to use alchemy on their weapons
\begin{center}
\rowcolors{2}{white}{gray}
\begin{tabularx}{\textwidth}{X X}
\hline 
Cost in equipment points &Effect\\
\hline 1 & Add element\\
3 & Add Ability from a Skill\\
6 & Add a level 1 skill \\
\hline
\end{tabularx}
\end{center}

\paragraph{Advanced Weapons}
Advancements may be purchased for any weapon.  There are 6 types of
advancements.  Advancements costs 15 points.  Each weapon may have 3
advancements.  The advancements may be of the same type, different types, or any
combination thereof.  The advancements do not affect the elemental affinity of
a weapon.

\begin{center}
\rowcolors{1}{white}{gray}
\begin{tabularx}{\textwidth}{X X}
\hline
Advancement & Effect\\
\hline
Earth & $+5$ to the attack stat\\
Sky & $+3$ to all initiative draws \\
Justice & $+1$ lv to all skills equipped to this weapon\\
Sea & Weapon gains and additional type attribute (short/long/melee/defense) or a second elemental affinity \\
Life & Gives the user $+1$ Will\\
Abandon & $+ 1$ to the card bonus ($card\, \geq15$ then the card is treated as a joker) \\
\hline
\end{tabularx}
\end{center}

\subsubsection{Hero definition}

All Player Characters and most computer controlled characters are Heroes.
Heroes have the following characteristics: Attributes: There are 4 attributes:
Strength, Speed, Defense, and Will. Strength, Speed, and Defense have a maximum value
of 5, and Will has a maximum value of 11. Strength, Speed, and Defense start at 0
and Will starts at 6.  Skills: a list of 5 skills that determine what abilities
may be placed in the abilities list.  Abilities: a list of all possible actions
that may be chosen from in a given battle.  Equipment: there are 5 equipment
slots: Helm, Chest, Foot, and 2 hands.  See the section on equipment for more
detail.  Elemental Affiliation: the element affiliated with a character.
Element is determined by the element of the hero s armor.  Info: non-essential
information such as name, description, etc.
\subsubsection{Equipment Definition}

Weapons have the following characteristics:
Attack: how much damage a weapon can do
Defense: how much damage a weapon can stop
Location: where the weapon is equipped: helm, foot, chest, hand, dual hand
Elemental Affiliation: what elemental type of damage a weapon does
Abilities: a specific ability that may be used.
Skills: Two slots for skills.  One slot may be accessed by engineering
(Knowledge Tree: Technical Skill) the other by enchanting (Magic Tree: Alchemy
Skill).  Weapons may also be purchased
/ acquired with skills in these slots.  Treat Skills equipped in this way a 
+1 level for the specific skill.
\subsection{Status System}
Status effects modify the attributes of a character.  There are seven types of status effects.  
\begin{itemize}
\item Buffs and Debuffs increase or decrease a statistic for a given duration.  
\item Recurring Buffs and Debuff do the same except the increase/
decrease is applied each round.
\item Add ability grants the use of a specific ability.
\item Remove ability prevents a specific ability or class of abilities.
\item Remove Equipment --- prevents the use of a piece of equipment until the duration has expired.
\item Prevent Status --- prevents the assignment of certain types of status effects
\item If Then Status --- When an allowed if then occurs then a status is imparted
\end{itemize}
